\section{Natural Numbers}

\subsection{Peano Postulates}

\begin{definition}{}{}
    Let $S : N \to N$ and let $A \subseteq N$. Then $A$ is said to be
    closed under $S$ iff
    \begin{equation*}
        S(x) \in A \text{ for all } x \in A
    \end{equation*}
\end{definition}

\begin{definition}{Peano System}{}
    Let $(N,S,e)$ be an ordered triple that consists of a set $N$, a
    function $S : N \to N$, and an element $e \in N$. Then $(N,S,e)$
    is a \textbf{Peano system} if the following conditions hold:
    \begin{enumerate}

        \item $e \notin ranS$
        \item $S$ is injective.
        \item For all $A \subseteq N$
            \begin{equation*}
                e \in A \land A \text{ is closed under } S \implies A = N
            \end{equation*}

    \end{enumerate}

\end{definition}

\subsection{Natural Number Set $\omega$}

\subsubsection{Inductive Sets}

In this section, we will construct natural numbers under the
architecture of set theory.

\begin{definition}{Successor}{}
    For each set $x$, the \textbf{successor} $x^{+}$ is the set defined by
    \begin{equation*}
        x^{+} = x \cup \{x\}
    \end{equation*}
\end{definition}

\begin{proposition}{}{}
    \begin{enumerate}

        \item $a \in x^{+} \iff a \in x \lor a = x$
        \item $x \in x^{+}$
        \item $x \subseteq x^{+}$

    \end{enumerate}
\end{proposition}

\begin{example}
    The first few natural numbers as follows
    \begin{itemize}

        \item $0 = \varnothing$
        \item $1 = \{0\}$
        \item $2 = \{0,1\}$
        \item $3 = \{0,1,2\}$
        \item $4 = \{0,1,2,3\}$

    \end{itemize}
\end{example}

\begin{definition}{Inductive}{}
    A set $I$ is said to be \textbf{inductive} iff
    \begin{enumerate}

        \item $\varnothing \in I$
        \item $\forall a \in I (a^{+} \in I)$

    \end{enumerate}
    The second one can be also restated as "The set is closed under successor."
\end{definition}

\begin{lemma}{Infinity Axiom}{}
    There is a inductive set.
    \begin{equation*}
        \exists I (\varnothing \in I \land \forall x \in I(x^{+} \in I))
    \end{equation*}
\end{lemma}

\begin{definition}{Natural Numbers}{}
    A \textbf{natural number} is an element that belongs to every
    inductive sets. In other words, $x$ is a natural number iff
    $x$ in every inductive sets.
\end{definition}

\begin{definition}{Natural Number Set}{}
    There exits a unique set $\omega$ such that for all $x$
    \begin{equation*}
        x \in \omega \iff x \text{ in every inductive set}
    \end{equation*}
    We denote the set as
    \begin{equation*}
        \omega = \{x : x \text{ in every inductive set}\}
    \end{equation*}
\end{definition}

\begin{proof}
    By infinity axiom, there exits an inductive set $A$. And
    \begin{equation*}
        x \text{ in every inductive set } \implies x \in A.
    \end{equation*}
    By \Cref{class}, there is a unique set $\omega$ s.t.
    \begin{equation*}
        x \in \omega \iff x \text{ in every inductive set}
    \end{equation*}
\end{proof}

\begin{theorem}{}{}
    The set $\omega$ is inductive and is a subset of every inductive
    set. Hence $\omega$ is the \textbf{smallest inductive set}.
\end{theorem}

\begin{corollary}{Principle of Mathematical Induction}{label:PMI}
    If $I$ is inductive and $I \subseteq \omega$, then $I = \omega$.
\end{corollary}

\begin{remarks}
    Suppose $P(n)$ is some property. To prove by induction that
    \begin{equation*}
        \forall n \in \omega P(n)
    \end{equation*}
    We just let $I = \{n \in \omega : P(n)\} \subseteq \omega$
    If we can prove that
    \begin{enumerate}

        \item $0 \in I$
        \item $n \in I \implies n^{+} \in I$

    \end{enumerate}
    Then $I = \omega$, by \Cref{label:PMI}. Therefore $P(n)$ holds
    for all natural numbers.
\end{remarks}

\begin{theorem}{}{}
    For every $n \in \omega$, either $n = 0$, or $n = k^{+}$ for
    some $k \in \omega$.
\end{theorem}

\begin{proof}
    Let $I = \{n \in \omega: n = 0 \lor \exists k \in \omega(n = k^{+})\}$\\
    \begin{itemize}

        \item Obviously, $0 \in I$.
        \item Let $n \in I$, then $n = 0 \lor n = k^{+} \text{ for some }
            k \in \omega$.
            \begin{enumerate}

                \item If $n = 0$, then $n^{+} = 0^{+}$, so $n^{+} \in I$.
                \item If $n = k^{+} \ \text{for some}\ k \in \omega$,
                    then $n^{+} =
                    (k^{+})^{+}$, so $n^{+} \in I$.

            \end{enumerate}

    \end{itemize}
    By \Cref{label:PMI}, $I = \omega$.
\end{proof}

\subsubsection{Transitive Sets}

There are some other important properties on natural numbers, such as

\begin{enumerate}

    \item $0 \in 1 \in 2 \in 3 \in 4 \in ...$
    \item $0 \subseteq 1 \subseteq 2 \subseteq 3 \subseteq 4 \subseteq ...$

\end{enumerate}

\begin{definition}{Transitive Sets}{}
    The set $A$ is called a \textbf{transitive set} if $\forall x \in
    A(x \subseteq A)$. In other words, every member of $A$ is also a
    subset of $A$.\\
    or more rigorously
    \begin{equation*}
        A \text{ is transitive } \iff (\forall a \in A)(\forall x \in
        a)(x \in A)
    \end{equation*}
\end{definition}

\begin{proposition}{}{}
    Let $A$ be a set
    \begin{equation*}
        A \text{ is transitive } \iff \bigcup A \subseteq A
    \end{equation*}
\end{proposition}

\begin{proposition}{}{}
    Let $A$ be a transitive set. Then
    \begin{equation*}
        \bigcup A^{+} = A
    \end{equation*}
\end{proposition}

\begin{proof}

\end{proof}

\begin{proposition}{}{}
    For all $m,n \in \omega$, if $n^{+} = m^{+}$, then $n = m$.
\end{proposition}

\begin{theorem}{}{}
    Every natural number is transitive set.
\end{theorem}

\begin{proof}
    We proof by induction. Let
    \begin{equation*}
        I = \{ n \in \omega : n \text{ is transitive}\}
    \end{equation*}
    Clearly, $0 \in I$. Let $n \in I$. We need to show that $n^{+} \in
    I$ and just $\bigcup n^{+} \subseteq n^{+}$.\\
    Firstly, $\bigcup n^{+} = n$, and $n \subseteq n^{+}$, Hence
    \begin{equation*}
        \bigcup n^{+} \subseteq n^{+}
    \end{equation*}
    By induction, $I = \omega$.
\end{proof}

\begin{theorem}{}{}
    The set $\omega$ is a transitive set.
\end{theorem}

\begin{proof}
    we shall prove $\forall n \in \omega(n \subseteq \omega)$ by induction.
\end{proof}

\subsection{Recursion Theorem on $\omega$}
