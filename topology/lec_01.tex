\section{Topological Spaces}

\subsection{Basic Concepts}\label{sub:Basic Concepts} % (fold)

\begin{definition}{Topological Spaces}{def:Topological Spaces}
    Let $X$ be a non-empty set and $\tau$ is a collection of subsets
    $U$ of $X$. If

    \begin{enumerate}

        \item $\varnothing, X \in \tau$
        \item $\bigcup_{\alpha} U_{\alpha} \in \tau$
        \item $\bigcap_{i} U_{i} \in \tau$

    \end{enumerate}
    Then we refer to $\tau$ as the \textbf{topology} on $X$, and the pair
    $(X,\tau)$ is called a \textbf{topological space}. For
    simplicity, we may just refer to $X$ itself as the
    topological space.
\end{definition}

\begin{remarks}
    $\tau$ is a subset of $\mathcal{P}(X)$, and we call the subset
    $U$ in $\tau$ the \textbf{open set}. So the three conditions
    above can be reclaimed by
    \begin{enumerate}
        \item $\varnothing$, $X$ are open.
        \item arbitrary union of open sets is open.
        \item finite intersection of open sets is open.
    \end{enumerate}
\end{remarks}

\begin{definition}{Discrete and Indiscrete Topology}{}
    Let $X$ be any non-empty set and $\tau$ be the collection of
    subsets of $X$.
    \begin{enumerate}

        \item If $\tau = \mathcal{P}(X)$, then $\tau$ is called the
            \textbf{discrete topology}, and $X$ is called a
            \textbf{discrete space}.
        \item If $\tau = \{\varnothing, X\}$, then $\tau$ is called the
            \textbf{indiscrete topology}, and $X$ is called a
            \textbf{indiscrete space}.

    \end{enumerate}
\end{definition}

\begin{proposition}{}{}
    Suppose $X$ is a topological space. If for every $x \in X$, the
    singleton $\{x\} \in \tau$, then $\tau$ is the
    discrete topology.
\end{proposition}

\begin{proof}
    Since $\tau \subseteq \mathcal{P}(X) $, we only need to show
    $\mathcal{P}(X) \subseteq \tau $.\\
    Take any subset $U \in \mathcal{P}(X) $, then $U$ can be expressed as
    \begin{equation*}
        U = \bigcup_{x \in S} \{x\}
    \end{equation*}
    Since every singletons are in $\tau$, so $U$ is also in $\tau$.
    Thus $\mathcal{P}(X) \subseteq \tau$.
\end{proof}

\begin{definition}{Closed Set}{}
    Let $X$ be a topological space. A subset $S$ of $X$ is said to be
    a \textbf{closed set}, if its complement, namely $X \setminus S$,
    is open in $X$.
\end{definition}

\begin{proposition}{}{}
    Let $X$ be a topological space, then
    \begin{enumerate}

        \item $\varnothing$, $X$ are closed.
        \item Finite union of closed sets is closed.
        \item Arbitrary intersection of closed sets is closed.

    \end{enumerate}
\end{proposition}

\begin{definition}{Clopen}{}
    A subset $S$ of topological space $X$ is called clopen if it is
    both open and closed.
\end{definition}

\begin{remarks}
    \begin{enumerate}

        \item In every topological space, both $\varnothing$ and
            $X$ are clopen.
        \item In the discrete space, all subsets of $X$ are clopen.
        \item In the indiscrete space, the only clopen sets are
            $\varnothing$ and $X$.

    \end{enumerate}
\end{remarks}

\begin{definition}{Cofinite Topology}{}
    Let $X$ be nonempty set. A topology $\tau$ on $X$ is called the
    \textbf{finite-closed topology} or the \textbf{cofinite topology}
    if the closed subsets are $X$ and all finite subsets of $X$. In
    other words, the open sets are $\varnothing$ and all subsets which
    have finite complements.
\end{definition}
% subsection Basic Concepts (end)

\subsection{Basis}\label{sub:Basis} % (fold)

\begin{definition}{Euclidean Topology on $\mathbb{R}$}{}
    A subset $S$ in $\mathbb{R}$ is said to be open if for each $x
    \in S$, there exit $a,b \in \mathbb{R}$, with $a < b$, s.t. $x
    \in (a,b) \subseteq S$. We refer to this kind of topology as the
    \textbf{euclidean topology on $\mathbb{R}$}.
\end{definition}

\begin{remarks}
    Whenever we refer to the topological space $\mathbb{R}$ without
    specifying the
    topology, we mean $\mathbb{R}$ with the euclidean topology.
\end{remarks}

\begin{example}
    For any $a,b \in \mathbb{R}$ and $a<b$, the interval $(a,b)$ is
    open, we refer to it as open interval.
\end{example}

\begin{example}
    For any $a,b \in \mathbb{R}$ and $a \le b$, the interval $[a,b]$ is closed,
    and it isn't open. When $a = b$, the interval $[a,b]$ just
    degenerate to the singleton $\{a\}$, which is also closed.
\end{example}

To describe the euclidean topology on $\mathbb{R}$ in more convenient
way, we introduce the notion of the basis for a topology.

\begin{proposition}{}{}
    A subset $S$ of $\mathbb{R}$ is open iff it is a union of open intervals.
\end{proposition}

\begin{proof}
    We have to show that:
    \begin{enumerate}

        \item if S is union of open intervals, then it is an open set.
        \item if S is an open set, then it is a union of open intervals.

    \end{enumerate}
    The first part is obvious. We only need to show the second part is true.\\
    Assume that $S$ is open in $\mathbb{R}$, then for each $x \in S$, there
    exits an interval $I_x = (a,b)$ s.t. $x \in I_x \subseteq S$.
    We claim that $S = \bigcup_{x \in S} I_x$. In order to show that
    two sets are equal, we need to prove that
    \begin{enumerate}

        \item $y \in S \implies y \in \bigcup_{x \in S} I_x$.
        \item $y \in \bigcup_{x \in S} I_x \implies y \in S$.

    \end{enumerate}
    Firstly, let $y \in S$, then $y \in I_y$. Obviously, $y \in
    \bigcup_{x \in S} I_x$.\\
    Secondly, let $y \in \bigcup_{x \in S} I_x$, then $y \in I_z$ for
    some $z \in S$. And $I_z \subseteq S$, Hence we conclude
    that $y \in S$.
\end{proof}

The above discussion tells us that in order to give the whole
topology of $\mathbb{R}$, it suffices to give just all open
intervals. Then all other open sets can be expressed as unions of
open intervals. This leads us to the concepts of basis.

\begin{definition}{Basis}{}
    Let $(X,\tau)$ be a topological space. A subcollection $\mathcal{B}$ of
    $\tau$ is said to be a \textbf{basis} for the
    topology $\tau$ if every open set is a union of elements of $\mathcal{B}$.
\end{definition}

\begin{remarks}
    It's obvious that $\mathcal{B} \subseteq \tau$, and there can be
    many different
    basis for the same topology.
\end{remarks}

\begin{example}
    Let $\mathcal{B} = \{(a,b): a,b \in \mathbb{R} \land a < b\}$.
    Then $\mathcal{B}$ is a basis for the euclidean topology on $\mathbb{R}$.
\end{example}

\begin{example}
    Let $\mathcal{B} = \{\{x\}: x \in \mathbb{R}\}$.
    Then $\mathcal{B}$ is a basis for the discrete topology on $\mathbb{R}$.
\end{example}

\begin{proposition}{}{}
    Let $(X,\tau)$ be a topological space. A subcollection $\mathcal{B}$ of
    $\tau$ is a basis for $\tau$ iff
    for any open set $U$, and any point $x \in U$, there is a $B \in
    \mathcal{B}$ s.t. $x \in B \subseteq U$.
\end{proposition}

\begin{proof}
    we need to show
    \begin{enumerate}

        \item if $\mathcal{B}$ is a basis of $\tau$, then for any
            open set $U$ and $x$, there exists $B \in \mathcal{B}$
            s.t. $x \in B \subseteq U$.
        \item if for any open set $U$ and $x$, there exits $B \in
            \mathcal{B}$ s.t. $x \in B \subseteq U$, then
            $\mathcal{B}$ is a basis of $\tau$.

    \end{enumerate}
    Firstly, assume $\mathcal{B}$ is a basis of $\tau$ and $U \in
    \tau$, then $U = \bigcup_{\alpha} B_{\alpha}$. So for any $x \in
    U$, there exits a $B_{t} \in \mathcal{B}$ s.t. $x \in B_{t} \subseteq U$.\\
    Conversely, for any open set $U$ and any $x \in U$,
    there exists a $B_{x} \in \mathcal{B}$ s.t. $x \in B \subseteq
    U$. We claim that $U = \bigcup_{x \in U} B_{x}$.
    \begin{itemize}
        \item $B_{x} \subseteq U \text{ for all } B_{x} \implies
            \bigcup_{x \in U} B_{x} \subseteq U$.
        \item $x \in U \implies x \in B_{x} \implies x \in \bigcup_{x
            \in U} B_{x}$.
    \end{itemize}
\end{proof}

\begin{proposition}{}{}
    Let $\mathcal{B}$ be a basis for a topology $\tau$ on a set $X$.
    Then a subset $U$ is open iff for each $x \in U$, there exist a
    $B \in \mathcal{B}$ s.t. $x \in B \subseteq U$.
\end{proposition}

\begin{proposition}{}{}
    Let $\mathcal{B}$ be a basis for a topology $\tau$ on a set $X$.
    Then a subset $U$ is open iff it is a union of elements of
    $\mathcal{B}$.
    \begin{align*}
        U \in \tau \iff U = \bigcup_{\alpha} B_{\alpha}
    \end{align*}
\end{proposition}
Sometimes we call it $\mathcal{B}$ generates the topology $\tau$.

\begin{theorem}{}{}
    Let $X$ be a non-empty set and let $\mathcal{B}$ be a collection
    of subsets of $X$. Then
    $\mathcal{B}$ is a basis for a topology on $X$ iff $\mathcal{B}$
    has the following properties:
    \begin{enumerate}
        \item $X = \bigcup_{B \in \mathcal{B}} B$.
        \item for any $B_{1},B_{2} \in \mathcal{B}$, the set $B_{1}
            \cap B_{2}$ is a union of elements of $\mathcal{B}$.
    \end{enumerate}
\end{theorem}

\begin{proof}
    If $\mathcal{B}$ is a basis for a topology $\tau$, then $\tau$
    satisfies the properties\ref{def:Topological Spaces}. Obviously
    $X$ and intersection of any two open sets are open sets. As the
    open sets are just the unions of elements of $\mathcal{B}$, it
    implies properties 1 and 2.\\
    Conversely, assume $\mathcal{B}$ has properties 1 and 2, and we will
    show that all unions of elements of $\mathcal{B}$ generate a
    topology $\tau$ which satisfies the
    properties\ref{def:Topological Spaces} and obviously
    $\mathcal{B}$ is a basis of $\tau$.
    \begin{enumerate}

        \item $\varnothing$, $X$ are unions of members of $\mathcal{B}$.
        \item Let $\{O_{\alpha}\}$ be a family of elements of $\tau$. Each
            $O_{\alpha}$ is a union of elements of $\mathcal{B}$, hence
            the union of $O_{\alpha}$ is also a union of elements of
            $\mathcal{B}$
        \item Let $T_{1}$ and $T_{2}$ be in $\tau$. $T_{1} =
            \bigcup_{\alpha} B_{\alpha}$ where
            $B_{\alpha} \in \mathcal{B}$, and $T_{2} =
            \bigcup_{\beta} B_{\beta}$ where
            $B_{\beta} \in \mathcal{B}$. Hence
            \begin{align*}
                T_{1} \cap T_{2} = \left(\bigcup_{\alpha}
                B_{\alpha}\right) \bigcap \left(\bigcup_{\beta}
                B_{\beta}\right) = \bigcup_{\alpha, \beta} \left(B_{\alpha}
                \cap B_{\beta}\right)
            \end{align*}
            By assumption, each $B_{\alpha} \cap B_{\beta}$ is a
            union of elements of $\mathcal{B}$, so is the $T_{1} \cap T_{2}$.
    \end{enumerate}

\end{proof}

\begin{theorem}{}{}
    Let $X$ be a non-empty set. $\mathcal{B}_{1}$ and
    $\mathcal{B}_{2}$ are bases for topology $\tau_{1}$ and
    $\tau_{2}$ respectively. Then $\tau_{1} = \tau_{2}$ iff
    \begin{enumerate}

        \item $\forall B \in \mathcal{B}_{1}, \forall x \in B,
            \exists B' \in \mathcal{B}_{2}$ s.t. $x \in B' \subseteq B$.
        \item $\forall  B'\in \mathcal{B}_{2}, \forall x \in B,
            \exists B \in \mathcal{B}_{1}$ s.t. $x \in B \subseteq B'$.

    \end{enumerate}
\end{theorem}

\begin{proof}

\end{proof}

\begin{example}
    Refer to $\{(x,y) \in \mathbb{R}^{2}: a<x<b \land
    c<y<d\}$ as the open rectangle and $\{(x,y) \in \mathbb{R}^{2}:
    (x-a)^2 + (y-b)^2 < \epsilon\}$ as the open disc. If
    $\mathcal{B}_{1}$ is the collection of all open rectangles and
    $\mathcal{B}_{2}$ is the collection of all open discs, then these
    two bases generate the same topology on $\mathbb{R}^2$ i.e. the
    euclidean topology on $\mathbb{R}^2$.
\end{example}
% section Basis (end)
\subsection{Limit Points}\label{sub:Limit Points} % (fold)

% section Limit Points (end)
