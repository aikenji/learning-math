\section{Topological Spaces}

\subsection{Basic Concepts}\label{sub:Basic Concepts} % (fold)

\begin{definition}{Topological Spaces}{}
    Let $X$ be a non-empty set and $\tau$ is a collection of subsets
    $U$ of $X$. If

    \begin{enumerate}

        \item $\varnothing, X \in \tau$
        \item $\bigcup_{\alpha} U_{\alpha} \in \tau$
        \item $\bigcap_{i} U_{i} \in \tau$

    \end{enumerate}
    Then we refer to $\tau$ as the \textbf{topology} on $X$, and the pair
    $(X,\tau)$ is called a \textbf{topological space}. For
    simplicity, we may just refer to $X$ itself as the
    topological space.
\end{definition}

\begin{remarks}
    $\tau$ is a subset of $\mathcal{P}(X)$, and we call the subset
    $U$ in $\tau$ the \textbf{open set}. So the three conditions
    above can be reclaimed by
    \begin{enumerate}
        \item $\varnothing$, $X$ are open.
        \item arbitrary union of open sets is open.
        \item finite intersection of open sets is open.
    \end{enumerate}
\end{remarks}

\begin{definition}{Discrete and Indiscrete Topology}{}
    Let $X$ be any non-empty set and $\tau$ be the collection of
    subsets of $X$.
    \begin{enumerate}

        \item If $\tau = \mathcal{P}(X)$, then $\tau$ is called the
            \textbf{discrete topology}, and $X$ is called a
            \textbf{discrete space}.
        \item If $\tau = \{\varnothing, X\}$, then $\tau$ is called the
            \textbf{indiscrete topology}, and $X$ is called a
            \textbf{indiscrete space}.

    \end{enumerate}
\end{definition}

\begin{proposition}{}{}
    Suppose $X$ is a topological space. If for every $x \in X$, the
    singleton $\{x\} \in \tau$, then $\tau$ is the
    discrete topology.
\end{proposition}

\begin{proof}
    Since $\tau \subseteq \mathcal{P}(X) $, we only need to show
    $\mathcal{P}(X) \subseteq \tau $.\\
    Take any subset $U \in \mathcal{P}(X) $, then $U$ can be expressed as
    \begin{equation}
        U = \bigcup_{x \in S} \{x\}
    \end{equation}
    Since every singletons are in $\tau$, so $U$ is also in $\tau$.
    Thus $\mathcal{P}(X) \subseteq \tau$.
\end{proof}

\begin{definition}{Closed Set}{}
    Let $X$ be a topological space. A subset $S$ of $X$ is said to be
    a \textbf{closed set}, if its complement, namely $X \setminus S$,
    is open in $X$.
\end{definition}

\begin{proposition}{}{}
    Let $X$ be a topological space, then
    \begin{enumerate}

        \item $\varnothing$, $X$ are closed.
        \item Finite union of closed sets is closed.
        \item Arbitrary intersection of closed sets is closed.

    \end{enumerate}
\end{proposition}

\begin{definition}{Clopen}{}
    A subset $S$ of topological space $X$ is called clopen if it is
    both open and closed.
\end{definition}

\begin{remarks}
    \begin{enumerate}

        \item In every topological space, both $\varnothing$ and
            $X$ are clopen.
        \item In the discrete space, all subsets of $X$ are clopen.
        \item In the indiscrete space, the only clopen sets are
            $\varnothing$ and $X$.

    \end{enumerate}
\end{remarks}

\begin{definition}{Cofinite Topology}{}
    Let $X$ be nonempty set. A topology $\tau$ on $X$ is called the
    \textbf{finite-closed topology} or the \textbf{cofinite topology}
    if the closed subsets are $X$ and all finite subsets of $X$. In
    other words, the open sets are $\varnothing$ and all subsets which
    have finite complements.
\end{definition}
% subsection Basic Concepts (end)

\subsection{The Euclidean Topology}\label{sub:The Euclidean Topology} % (fold)

\begin{definition}{Euclidean Topology on $\mathbb{R}$}{}
    A subset $S$ in $\mathbb{R}$ is said to be open if for each $x
    \in S$, there exit $a,b \in \mathbb{R}$, with $a < b$, s.t. $x
    \in (a,b) \subseteq S$. We refer to this kind of topology as the
    \textbf{euclidean topology on $\mathbb{R}$}.
\end{definition}

\begin{remarks}
    Whenever we refer to the topological space $\mathbb{R}$ without
    specifying the
    topology, we mean $\mathbb{R}$ with the euclidean topology.
\end{remarks}

To describe the euclidean topology on $\mathbb{R}$ in more convenient
way, we introduce the notion of the basis for a topology.

\begin{proposition}{}{}
    A subset $S$ of $\mathbb{R}$ is open iff it is a union of open intervals.
\end{proposition}

\begin{proof}
    We have to show that:
    \begin{enumerate}

        \item if S is union of open intervals, then it is an open set.
        \item if S is an open set, then it is a union of open intervals.

    \end{enumerate}
    The first part is obvious. We only need to show the second part is true.\\
    Assume that $S$ is open in $\mathbb{R}$, then for each $x \in S$, there
    exits an interval $I_x = (a,b)$ s.t. $x \in I_x \subseteq S$.
    We claim that $S = \bigcup_{x \in S} I_x$. In order to show that
    two sets are equal, we need to prove that
    \begin{enumerate}

        \item $y \in S \implies y \in \bigcup_{x \in S} I_x$.
        \item $y \in \bigcup_{x \in S} I_x \implies y \in S$.

    \end{enumerate}
    Firstly, let $y \in S$, then $y \in I_y$. Obviously, $y \in
    \bigcup_{x \in S} I_x$.\\
    Secondly, let $y \in \bigcup_{x \in S} I_x$, then $y \in I_z$ for
    some $z \in S$. And $I_z \subseteq S$, Hence we conclude
    that $y \in S$.
\end{proof}

The above discussion tells us that in order to give the whole
topology of $\mathbb{R}$, it suffices to give just all open
intervals. Then all other open sets can be expressed as unions of
open intervals. This leads us to the concepts of basis.

\begin{definition}{Basis}{}
    Let $(X,\tau)$ be a topological space. A collection $\mathcal{B}$ of
    open subsets of $X$ is said to be a \textbf{basis} for the
    topology $\tau$ if every open set is a union of elements of $\mathcal{B}$.
\end{definition}

\begin{example}
    Let $\mathcal{B} = \{(a,b): a,b \in \mathbb{R} \land a < b\}$.
    Then $\mathcal{B}$ is a basis for the euclidean topology on $\mathbb{R}$.
\end{example}
% subsection The Euclidean Topology (end)
